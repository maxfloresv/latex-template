\newcommand*\ALLR{\Longleftrightarrow\quad}
\newcommand*\ALR{\Longrightarrow\quad}

\renewcommand{\epsilon}{\varepsilon}

\newcommand{\ans}[1]{{\color{red!50!black} #1}}
\newcommand{\asignar}[1]{$\mathbf{#1}$ \textbf{pts.}}
\newcommand{\obs}[1]{\textit{(Obs.: #1)}}

% Esto genera el título del documento. Formato 
% #1 Describe al documento (título general)
% #2 Nombre del ramo
% #3 Número de sección
% #4 Mail del aux
\newcommand{\setup}[4]{%
    \begin{minipage}[t]{0.7\linewidth}
    \vspace{-1.5cm}
        {\large\textbf{#1}}\par 
        \textbf{#2. Sección #3.}\par
        {Mail: \texttt{#4}}
    \end{minipage}
    \hfill
    \begin{minipage}[t]{0.2\linewidth}
        \includegraphics[width=\linewidth,right]{img/fcfm.pdf}
    \end{minipage}
}

\newcommand{\defp}[1]{[\textsc{#1}]}

\newcommand{\resumen}[2]{\par
    \textbullet \hspace{0.5mm} \defp{#1} #2
\par}

% Destacar expresiones matemáticas
\newcommand{\hmath}[1]{\tcbhighmath[
    left=.3pt, right=.3pt, top=.3pt, bottom=.3pt,
    colframe=white, colback=blue!10!white
]{#1}}
\newcommand{\emath}[1]{\tcbhighmath[
    left=.3pt, right=.3pt, top=.3pt, bottom=.3pt,
    colframe=white, colback=red!20!white
]{#1}}

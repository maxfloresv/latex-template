\setlength{\parindent}{0pt}
\setlength{\parskip}{3mm}
\setlength{\itemsep}{3mm}

% Colores: Se pueden sacar de esta página los códigos https://www.rapidtables.com/web/color/RGB_Color.html
\definecolor{mygray}{RGB}{41, 48, 52}
\definecolor{textind}{RGB}{107, 33, 41}
\definecolor{mathind}{RGB}{61, 31, 34}
\definecolor{golden}{RGB}{255, 215, 0}
\definecolor{darkgold}{RGB}{170, 108, 57}

% Cambiar los valores dentro de las llaves para editar la metadata del PDF
\hypersetup{pdfinfo={
                Title={Titulo del PDF},
                Author={Autor del PDF}
            }}

% Tipo de letra (acá se pueden buscar más: https://tug.org/FontCatalogue/mathfonts.html)
\usepackage{lmodern}
\usepackage[T1]{fontenc}     

% La opción parbox en false permite que parskip efectúe cambios en el espaciado: https://tex.stackexchange.com/a/154994/248558
\tcbset{parbox=false}

% Importante para que las ecuaciones alineadas con el entorno alignat* SÍ se corten en cambios de página.
\allowdisplaybreaks

% Lo que aparece en el header y footer:
\fancypagestyle{normal}{
    \fancyhf{}
    \fancyhead[LO]{\small \textrm{Facultad de Ciencias Físicas y Matemáticas}}
    \fancyhead[RO]{\small \textrm{Universidad de Chile}}
    \fancyfoot[CO]{\textrm{---\ \thepage\ ---}}
}

% Establece todas las ecuaciones al formato "displaystyle" (quitar % para ver cambios)
% \everymath{\displaystyle}
